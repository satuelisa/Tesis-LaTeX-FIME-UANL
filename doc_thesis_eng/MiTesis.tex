%%%%%%%%%%%%%%%%%%%%%
% Documento maestro %
%%%%%%%%%%%%%%%%%%%%%
\documentclass{fime_eng}

%%%%%%%%%%%%%%%%%%%%%%%%%%%%%%%%%%%%%%%%%%%
% Cargando paquetes y definiendo opciones %
%%%%%%%%%%%%%%%%%%%%%%%%%%%%%%%%%%%%%%%%%%%
% Aquí puedes cargar los paquetes que vas a usar. La clase
% fime ya incluye babel, inputenc, graphicx y los de la AMS.
% Cargar un paquete está a tu libertad (y responsabilidad).
\usepackage{hyperref}
    \hypersetup{breaklinks=true,colorlinks=true,linkcolor=black,citecolor=black,urlcolor=black}
\usepackage[spanish,textsize=scriptsize]{todonotes}
\usepackage{multicol}

%%%%%%%%%%%%%%%%%%%%%
% Definiendo campos %
%%%%%%%%%%%%%%%%%%%%%
\def\titulo{Título en español}
\def\tituloEng{Title in english}
\def\autor{Nombre Apellidos}
\def\matricula{1234567}
\def\grado{Doctorado en\ldots}
\def\gradoEng{Doctor in \ldots}
% En caso de que el grado tenga orientación o especialidad llenar el siguiente
% campo dejando un ESPACIO INICIAL, en caso contrario, dejar vacío
\def\orientacion{}
% Coloca el mes con mayúscula inicial
\def\fecha{Mes 20XX}
\def\fechaEng{Month 20XX}

\def\asesor{Grado.\ Nombre Apellidos}
\def\asesorEng{Nombre Apellidos, Grado.}
\def\revisorA{Grado.\ Nombre Apellidos}
\def\revisorAEng{Nombre Apellidos, Grado.}
\def\revisorB{Dra.\ Nombre Apellidos}
\def\revisorBEng{Nombre Apellidos, Grado.}
% En el caso de que tu tesis sea de doctorado activa la variable cambiándola a \doctoradotrue
% y define tus otros dos revisores
\newif\ifdoctorado\doctoradotrue
\def\revisorC{Grado.\ Nombre Apellidos}
\def\revisorCEng{Nombre Apellido, Grado.}
\def\revisorD{Grado.\ Nombre Apellido}
\def\revisorDEng{Nombre Apellido, Grado.}
% El visto bueno siempre va
\def\vobo{Grado. Nombre Apellido}
\def\voboEng{Nombre Apellido, Grado.}

%%%%%%%%%%%%%%%%%%%%%%%
% Inicia el documento %
%%%%%%%%%%%%%%%%%%%%%%%
\begin{document}

\frontmatter
\pagestyle{main}

\include{Chapters/0Portadas-spa}
% Dedicatoria

\thispagestyle{empty}
\vspace*{17mm}

\begin{flushright}
\begin{itshape}

Algo breve, como un minicuento.

\end{itshape}
\end{flushright}



\tableofcontents
\listoffigures
\listoftables

%Agradecimientos

\chapter{Agradecimientos}
\markboth{Agradecimientos}{}

A FIME, UANL, CONACYT y demás.

%Resumen

\chapter{Resumen}
\markboth{Resumen}{}

{\renewcommand{\baselinestretch}{1.1}\selectfont
{\setlength{\leftskip}{10mm}
\setlength{\parindent}{-10mm}

\autor.

Candidato para obtener el grado de \grado\orientacion.

\uanl.

\fime.

Título del estudio: \textsc{\titulo}.

\noindent Número de páginas: \pageref*{lastpage}.}

\paragraph{Introducción}
Lorem ipsum.

\paragraph{Antecedentes}
Lorem ipsum.

\paragraph{Hipótesis}
Lorem ipsum.

\paragraph{Preguntas de investigación}
\begin{itemize}
    \item ¿Pregunta 1?
    \item ¿Pregunta 2?
    \item ¿Pregunta 3?
\end{itemize}

\paragraph{Objetivo general}
Lorem ipsum.

\paragraph{Objetivos específicos}
\begin{itemize}
    \item Objetivo 1.
    \item Objetivo 2.
    \item Objetivo 3.
\end{itemize}

\paragraph{Método de estudio}
Lorem ipsum.

\paragraph{Contribuciones y conclusiones}
Lorem ipsum.

\paragraph{Trabajo a futuro}
Lorem ipsum.

\paragraph{Estructura de la tesis}
Lorem ipsum.

\vspace{15mm}

Firmas de asesores

\begin{tabular}{p{35mm}p{20mm}p{12mm}p{20mm}p{35mm}}
	\cline{1-2} \cline{4-5} \\ [-3mm]
	\multicolumn{2}{c}{\asesor} & & \multicolumn{2}{c}{\revisorA} \\
	\multicolumn{2}{c}{Coasesora}  & & \multicolumn{2}{c}{Coasesor} \\ [13mm]
\end{tabular}
%Resumen

\chapter{Abstract}
\markboth{Abstract}{}

{\renewcommand{\baselinestretch}{1.1}\selectfont
{\setlength{\leftskip}{10mm}
\setlength{\parindent}{-10mm}

\autor.

As a candidate to obtain the degree of \gradoEng\orientacion.

\uanl.

\fime.

Title of the study: \textsc{\tituloEng}.

\noindent Number of pages: \pageref*{lastpage}.}

\paragraph{Introduction}
Lorem ipsum

\paragraph{Background}
Lorem ipsum

\paragraph{Hipótesis}
Lorem ipsum

\paragraph{Research questions}
\begin{itemize}
    \item Question 1?
    \item Question 2?
    \item Question 3?
\end{itemize}

\paragraph{General objective}
Lorem ipsum.

\paragraph{Specific objectives}
\begin{itemize}
    \item Objective 1.
    \item Objective 2.
    \item Objective 3.
\end{itemize}

\paragraph{Methodology}
Lorem ipsum.

\paragraph{Contributions and conclusions}
Lorem ipsum.

\paragraph{Future work}
Lorem ipsum.

\paragraph{Thesis structure}
Lorem ipsum.

\vspace{15mm}

\begin{tabular}{p{35mm}p{20mm}p{12mm}p{20mm}p{35mm}}
	\cline{1-2} \cline{4-5} \\ [-3mm]
	\multicolumn{2}{c}{\asesorEng} & & \multicolumn{2}{c}{\revisorAEng} \\
	\multicolumn{2}{c}{Advisor}  & & \multicolumn{2}{c}{Advisor} \\ [13mm]
\end{tabular}

\mainmatter
\pagestyle{fime}

\chapter{Introduction}
\label{ch_Intro}

Lorem ipsum.
\chapter{Related literature}
\label{ch_Rel}

Lorem ipsum.
\chapter{Theoretical basis}
\label{ch_The}

Lorem ipsum.
\chapter{Methodology}
\label{ch_Met}

Lorem ipsum.
\chapter{Results}
\label{ch_Res}

Lorem ipsum.
\chapter{Conclusions}
\label{ch_Con}

Lorem ipsum.

\appendix
%%% Desbloquear si tienes apéndice
%\include{Apendice1}

\backmatter
\pagestyle{main}

% Aquí va la bibliografía, puedes usar el entorno de LaTeX (thebibliography)
% o la herramienta BibTeX. En caso de que optes por BibTeX, puedes usar
% alguno de los archivos de estilo (mighelbib.bst o mighelnat.bst) incluidos
% en el paquete, cuyos diseños armonizan con el diseño de tesis provisto por
% fime.cls. Para muestra, basta un botón:
\bibliographystyle{plainnat}
\bibliography{MiBiblio}

\label{lastpage}
%Autobiografia

\chapter*{Resumen autobiográfico}
\thispagestyle{empty}

\begin{center}
\autor

Candidato para obtener el grado de\\
\grado\\
\orientacion\bigskip

\uanl\\
\fime\bigskip

Tesis:\\
\textsc{\large\titulo}
\end{center}\bigskip

%Aquí va tu historia
Nací el X de mes de año en Ciudad, País; mis padres son Nombres Apellidos y Nombres Apellidos. En año egresé de la carrera de X en la Facultad de X de la Universidad X. En año me gradué como Maestro en X en la Facultad X de la Universidad X.

\end{document}

