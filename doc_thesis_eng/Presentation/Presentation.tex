\documentclass[9pt]{beamer}
\usetheme{AnnArbor} %Darmstadt %Warsaw %AnnArbor 
\usecolortheme{spruce} %seahorse 
\usepackage[spanish,mexico]{babel}
\usepackage[utf8]{inputenc}
\usepackage{graphicx}
\usepackage{times}
\usefonttheme{structurebold}
\setbeamercovered{transparent} %transparent %dynamic
\usepackage{subfigure}
\usepackage{amssymb, amsmath, amsbsy}

% https://tex.stackexchange.com/a/502353
\usepackage{enumitem}
\newlist{todolist}{itemize}{2}
\setlist[todolist]{label=$\square$}
\usepackage{pifont}
\newcommand{\cmark}{\ding{51}}%
\newcommand{\done}{\rlap{$\square$}{\raisebox{2pt}{\large\hspace{1pt}\cmark}}%
\hspace{-2.5pt}}

\usepackage[sort&compress,numbers]{natbib}
\selectlanguage{spanish}
\usepackage[spanish,onelanguage]{algorithm2e}
\usepackage{listings}
\newcounter{saveenumi}
\newcommand{\seti}{\setcounter{saveenumi}{\value{enumi}}}
\newcommand{\conti}{\setcounter{enumi}{\value{saveenumi}}}
\usepackage{tabu}
\extrarowsep = 1mm
\usepackage{multirow}
\usepackage[export]{adjustbox}

\usepackage{soul} % Para tachar texto

\defbeamertemplate*{title page}{propio}[1][]
{
\vbox{}
\vfill
\begingroup
\centering    
{\usebeamercolor[fg]{titlegraphic}\inserttitlegraphic\par}
\vspace{0.1 \paperheight}%
\begin{beamercolorbox}[sep=8pt,center,#1]{title}
	\usebeamerfont{title}\inserttitle\par%
	\ifx\insertsubtitle\@empty%
	\else%
	\vskip0.25em%
	{\usebeamerfont{subtitle}\usebeamercolor[fg]{subtitle}\insertsubtitle\par}%
	\fi%     
\end{beamercolorbox}%
\vskip1em\par
\begin{beamercolorbox}[sep=8pt,center,#1]{author}
	\usebeamerfont{author}\insertauthor
\end{beamercolorbox}
\begin{beamercolorbox}[sep=8pt,center,#1]{institute}
	\usebeamerfont{institute}\insertinstitute
\end{beamercolorbox}
\begin{beamercolorbox}[sep=8pt,center,#1]{date}
	\usebeamerfont{date}\insertdate
\end{beamercolorbox}\vskip0.5em

\endgroup
\vfill
}

%https://ondahostil.wordpress.com/2016/11/15/lo-que-he-aprendido-esquemas-en-latex/
\usepackage{schemata}
\newcommand\diagram[2]{\schema{\schemabox{#1}}{\schemabox{#2}}}
\setbeamertemplate{navigation symbols}{}

\graphicspath{ {../Figuras/} }  % Las figuras se colocan en la carpeta Figuras
%Recuerde que para agregar la carpeta de figuras se debe ejecutar las funciones siguientes: pdflatex bibtex pdflatex pdflatex (F6+F11+F6+F6) desde línea de comandos o desde la interfaz que usted utilice. 

\title[Título corto]{Título súper largo para evidenciar que el pie de diapositiva muestra el Título corto}
\author{{\bf Nombres Apellidos}}
\institute[PISIS FIME UANL]{\sc Posgrado en Ingeniería con especialidad en Ingeniería de Sistemas \\
Facultad de Ingeniería Mecánica y Eléctrica\\
Universidad Autónoma de Nuevo León}
\date{\today}

\titlegraphic{
%\includegraphics[scale=1.0]{UANL.png}\hspace*{5cm}
\includegraphics[scale=0.8]{uanl.eps}\hspace*{4cm}
\includegraphics[scale=0.4]{fime.eps}
}


\AtBeginSection[]{\frame{\frametitle{}\tableofcontents[currentsection]}}

\begin{document}
\frame
\titlepage

\begin{frame}{Contenido}
	\tableofcontents
\end{frame}


\section{Introducción}
\begin{frame}{Título de la diapositiva}
	\begin{columns}
		\begin{column}{0.45\textwidth}
			\begin{block}{Bloque}
				Ejemplo de bloques en columnas
			\end{block}

			\begin{block}{Bloque}
				\begin{itemize}
					\item Elemento 1.
					\item Elemento 2.
					\item Elemento 3.
				\end{itemize}
			\end{block}
			
		\end{column}
		\begin{column}{0.45\textwidth}
			\begin{block}{Bloque}
				Ejemplo de bloques en columnas
			\end{block}

			\begin{block}{Bloque}
				Ejemplo de bloques en columnas
			\end{block}
		\end{column}
	\end{columns}
\end{frame}

\begin{frame}{Hipótesis}
	Lorem ipsum.
\end{frame}

\begin{frame}{Objetivos}
	\begin{itemize}
		\item Objetivo 1.
		\item Objetivo 2.
		\item Objetivo 3.
	\end{itemize}
\end{frame}

\section{Antecedentes}
\subsection{Estudios}
\begin{frame}{Título}
	Lorem ipsum
\end{frame}

\subsection{Motivación}
\begin{frame}{Título}
	Lorem ipsum
\end{frame}


\section{Metodología y resultados}
\subsection{Subsección 1}
\begin{frame}{Título}
	Lorem ipsum
\end{frame}

\subsection{Subsección 2}
\begin{frame}{Título}
	Lorem ipsum
\end{frame}

\subsection{Subsección 3}
\begin{frame}{Título}
	Lorem ipsum
\end{frame}

\section{Resultados}
\begin{frame}{Título}
	Lorem ipsum
\end{frame}

\section{Conclusiones y contribuciones}
\begin{frame}{Conclusiones}
	\begin{itemize}
		\item Conclusión 1.
		\item Conclusión 2.
		\item Conclusión 3.
	\end{itemize}
\end{frame}

\begin{frame}{Contribuciones}
	\begin{itemize}
		\item Contribución 1.
		\item Contribución 2.
		\item Contribución 3.
	\end{itemize}
\end{frame}

\begin{frame}{Objetivos}
	\begin{todolist}
		\item[\done] Objetivo cumplido.
		\item Objetivo faltante.
	\end{todolist}
\end{frame}

\section{Trabajo a futuro}
\begin{frame}{Trabajo a futuro}
	\begin{itemize}
		\item Futuro 1.
		\item Futuro 2.
		\item Futuro 3.
	\end{itemize}
\end{frame}

\section*{Agradecimientos}
\begin{frame}
	\center{Gracias por su atención.} \\
	
	\url{usuario@correo.dominio}
	
\end{frame}


\section*{Bibliografía}
\begin{frame}[allowframebreaks]
	\bibliographystyle{apalike}
	\bibliography{../MiBiblio.bib}
\end{frame}

\section{Anexos}
\begin{frame}{Título}
	Lorem ipsum
\end{frame}

\end{document}